\documentclass[a4paper, 12pt]{report}

%%%%%%%%%%%%
% Packages %
%%%%%%%%%%%%

\usepackage[french]{babel}
\usepackage[noheader]{packages/sleek}
\usepackage{packages/sleek-title}
\usepackage[french]{packages/sleek-theorems}
\usepackage{packages/sleek-listings}
\usepackage{tikz}
\usepackage[french,linesnumbered,lined]{algorithm2e}
\usepackage{movie15}
\usepackage{graphicx}

\SetKwInput{KwResult}{R\'esultat}
\SetKw{KwInput}{Entr\'ees}
%%%%%%%%%%%%%%
% Title-page %
%%%%%%%%%%%%%%

\logo{./resources/img/up_maths-info.jpg}
\institute{Université de Paris}
\faculty{UFR Mathématiques et Informatique}
\title{Projet TER - StyleGANov'}
\subtitle{Master 1 Vision Machine Intelligente}
%\subtitle{Master 1 Vision et Machine Intelligente}
\author{Prénom \textsc{Jiaxin XUE} -- Prénom \textsc{Billa IHADDADEN}\\
{\small Encadré par Prénom \textsc{Olivier RISSER-MAROIX}}}
\date{Année universitaire 2020-2021}

%%%%%%%%%%%%%%%%https://www.overleaf.com/project/6012bd305aee4065605e8f1d
% Bibliography %
%%%%%%%%%%%%%%%%

\addbibresource{./resources/bib/references.bib}

%%%%%%%%%%
% Others %
%%%%%%%%%%

\lstdefinestyle{latex}{
    language=TeX,
    style=default,
    %%%%%
    commentstyle=\ForestGreen,
    keywordstyle=\TrueBlue,
    stringstyle=\VeronicaPurple,
    emphstyle=\TrueBlue,
    %%%%%
    emph={LaTeX, usepackage, textit, textbf, textsc}
}

\FrameTBStyle{latex}

\def\tbs{\textbackslash}

%%%%%%%%%%%%
% Document %
%%%%%%%%%%%%

\begin{document}
    \maketitle

    \romantableofcontents

    
    \chapter{Introduction}

Le jugement de la similarité d’images par l’humain s’appuie sur beaucoup de choses notamment les éléments de la scène et les aspects culturels. Pour nous les humains il nous est facile de juger de la similarité entre deux images, cependant la prédiction de la similarité perceptive humaine est un sujet de recherche difficile. Le processus visuel sous-jacent à cet aspect de la vision humaine fait appel à plusieurs niveaux différents d’analyse visuelle (formes, objets, texture, disposition, couleur…etc). Dans le cas de ce projet la similarité purement visuelle sera traitée sans prendre en compte la sémantique. Le transfert de style et la génération d’image par des architectures modernes de réseaux de neurones (VAE, ADalN, PIX2PIX, CycleGAN, BigGAN…etc). La première partie consiste à expérimenter la génération d’image avec pix2pix en utilisant les images de l’ensemble de données PASCAL avec et sans l’arrière-plan. Ensuite Nous allons expérimenter le BigGAN pré entrainé sur l’ensemble de données d’ImageNet.
    % in documenet
    %\includemovie{2cm}{2cm}{cat2dog.gif}
    %\includegraphics{cat2dog.gif}

    \chapter{État de l'art}
    Un chapitre dédié à l'état de l'art doit décrire les concepts et méthodes déjà existant(e)s, en lien avec le travail que vous avez réalisé.
    
    \chapter{Contribution}
    Ce chapitre doit présenter les concepts et méthodes que vous avez étudié(e)s, et décrire vos résultats. En particulier, dans le cadre d'une étude expérimentale, discutez vos résultats et comparez, le cas échéant, votre approche avec l'état de l'art.

    \chapter{Rédaction en \LaTeX}
    Dans ce chapitre, nous décrivons quelques bases sur l'utilisation de \LaTeX{} pour la rédaction de votre rapport.
    Si vous êtes débutant en \LaTeX{}, de nombreux tutoriaux sont disponibles en ligne (notamment \href{https://www.overleaf.com/learn}{ce tutorial disponible sur Overleaf}). De nombreux symboles mathématiques sont disponbles \LaTeX{}. Certains sont présentés dans ce document, nous vous recommandons également d'utiliser \enquote{The Comprehensive \LaTeX{} Symbol List} \cite{pakin2020comprehensive} pour trouver les symboles dont vous avez besoin.
    
    Ce template est basé sur \href{https://fr.overleaf.com/latex/templates/sleek-template/hrksrrdywhfk}{Sleek Template}, un template \LaTeX{} libre qui fournit un certain nombre de commandes pour faciliter la rédaction d'un rapport. Nous décrivons ici une partie qui doit être suffisante, mais vous pouvez aussi vous référer à la documentation en ligne pour découvrir plus de commandes.
    
    
    Le format de la page de garde est généré automatiquement, en fonction de la valeur placée dans les champs suivants : \verb+\logo+, \verb+\institute+, \verb+\faculty+, \verb+\department+, \verb+\title+, \verb+\subtitle+, \verb+\author+, et \verb+\date+.

    Pensez à modifier \verb+\title+ et \verb+\author+, et à choisir le bon \verb+\subtitle+ en fonction de votre parcours (IAD ou VMA). Ne modifiez pas les autres champs.

    \begin{lstlisting}[style=latexFrameTB, caption={Code \LaTeX{} de la page de garde},gobble=7]
       \logo{./resources/img/up_maths-info.jpg}
       \institute{Université de Paris}
       \faculty{UFR Mathématiques et Informatique}
       \title{Titre du rapport}
       \subtitle{Master 1 Intelligence Artificielle Distribuée}
       %\subtitle{Master 1 Vision et Machine Intelligente}
       \author{Prénom \textsc{Nom1} -- Prénom \textsc{Nom2}}
       \date{2021}
    \end{lstlisting}

    
    \section{Découpage d'un chapitre}
    Au sein d'un chapitre, les commandes \verb+\section+, \verb+\subsection+ et \verb+subsubsection+, notamment, permettent d'organiser le contenu. De nombreux éléments (notamment les sections, les algorithmes, les définitions,$\dots$) peuvent être étiquetés au moyen de la commande \verb+\label+. Cela permet d'y faire référence, par exemple \verb+\ref{section:liste}+ permet de faire référence à la Section~\ref{section:liste}.
    
    \section{Listes}\label{section:liste}
    Cette section décrit comment réaliser des listes (et donne un exemple de section découpée en sous-sections). Pour chaque type de liste, les éléments sont identifiés par la commande \verb+\item+. Le label peut être modifié, soit individuellement avec \verb+\item[newLabel]+, soit pour tout l'environnement avec l'option \verb+label=newLabel+.
    
    
    \subsection{Listes à puces}
    Une liste à puce est définie via l'environnement \verb+itemize+ :
    \begin{itemize}
        \item un item,
        \item un autre item,
        \item[$*$] un dernier item, avec un label modifié.
    \end{itemize}
    
    \subsection{Listes numérotées}
    Les listes numérotées sont définies avec \verb+enumerate+ :
    \begin{enumerate}[label=(\alph*)]
        \item un item,
        \item un autre item,
        \item un dernier item.
    \end{enumerate}
      Dans ce cas, le nouveau label peut être une expression spéciale (\emph{cf.} Table~\ref{tab:enumerate_special_expressions} en Annexe), comme dans l'exemple précédent qui remplace la numérotation standard par \enquote{(a), (b), (c), ...}.
    
    \subsection{Listes descriptives}
    Une liste descriptive est définie par \verb+description+ :
    \begin{description}
        \item[Item 1] Description de l'item 1,
        \item[Item 2] Description de l'item 2,
        \item[Item 3] Description de l'item 3.
    \end{description}

    \subsection{Listes imbriquées}
    Voici un exemple d'imbrication de listes, qui inclut aussi des personnalisations, et des éléments de mathématiques (voir la Section~\ref{section:maths}).
    
    It is also possible to write nested lists. Here follows a very condensed example.

    \begin{itemize}
        \item Lorem ipsum dolor sit amet, consectetur adipiscing elit, sed do eiusmod tempor incididunt ut labore et dolore magna aliqua.

        Arcu ac tortor dignissim convallis aenean et tortor. In eu mi bibendum neque egestas congue quisque.

        \item[$+$] Semper quis lectus nulla at volutpat diam ut. Felis eget velit aliquet sagittis id. Blandit aliquam etiam erat velit scelerisque in dictum non consectetur.
        \begin{equation}\label{equation:exemple}
            a^2 + b^2 = c^2
        \end{equation}

        \item Nibh sed pulvinar proin gravida hendrerit lectus. Pretium aenean pharetra magna ac placerat vestibulum lectus mauris. Non consectetur a erat nam at lectus urna duis.
        \begin{enumerate}[noitemsep, label=\roman*.]
            \item Nibh tortor id aliquet lectus. Sit amet justo donec enim diam vulputate ut pharetra sit.
            \setcounter{enumi}{3}
            \item Condimentum id venenatis a condimentum vitae. Quis eleifend quam adipiscing vitae proin sagittis nisl.
            \addtocounter{enumi}{15}
            \item Proin sagittis nisl rhoncus mattis rhoncus urna neque viverra.
        \end{enumerate}

        \item Elit scelerisque mauris pellentesque pulvinar pellentesque habitant morbi tristique senectus.
            \begin{description}
                \item[Ridiculus] mus mauris vitae ultricies leo. Mollis aliquam ut porttitor leo a diam. Velit egestas dui id ornare arcu odio ut sem nulla.
                \item[Nullam vehicula] ipsum a arcu. Nibh sit amet commodo nulla facilisi nullam. At erat pellentesque adipiscing commodo elit. Libero volutpat sed cras ornare arcu dui.
            \end{description}
    \end{itemize}

    \section{Mathématiques}\label{section:maths}
    Ce template utilise les packages \texttt{amsmath} et \texttt{amssymb}, qui sont standard pour l'écriture d'éléments mathématiques en \LaTeX{}. Quelques commandes personnalisées ont également été définies, comme \verb+\rbk+, \verb+sbk+ et \verb+\cbk+ pour les parenthèses, les crochets et les accolades, \verb+\abs+ pour la valeur absolue, \verb+\norm+ pour la norme, et \verb+\fact+ pour la factorielle. En voici des exemples
    $$
        \rbk{\frac{\pi}{2}}, \quad \sbk{\frac{\pi}{2}}, \quad \cbk{\frac{\pi}{2}}, \quad \abs{\frac{\pi}{2}}, \quad \norm{\frac{\pi}{2}}, \quad \fact{n} = \prod_{i = 1}^{n} i
    $$
    
    On peut faire référence à une équation ou une formule mathématique particulière, comme dans l'exemple de l'équation~\ref{equation:exemple}.


    

   

    Voici quelques exemples qui illustrent les possibilités du template.

D'abord, une équation non numérotée avec l'environnement \verb+equation*+ :
    \begin{equation*}
        e = \sum_{n=0}^\infty \frac{1}{n!}
    \end{equation*}

Deux sous-équations avec leur propre label (\ref{equation:exemple-a} et \ref{equation:exemple-b}) :
    \begin{subequations}
        \begin{align}
            \frac{\diff x}{\diff t} & = \alpha x - \beta xy\label{equation:exemple-a} \\
            \frac{\diff y}{\diff t} & = \delta xy - \gamma y\label{equation:exemple-b}
        \end{align}
    \end{subequations}

    Un système d'équations :
    \begin{equation}
        \left\{
        \begin{aligned}
            x & = 2 \times y + z \\
            y & = 3 \times x^2 - 2 \times z \\
            z & = x - y
        \end{aligned}
        \right.
    \end{equation}


    \section{Figures}
    \subsection{Inclusion d'images}
    Les figures, aux formats classiques d'images, peuvent être incluses avec \verb+\includegraphics{}+, notamment \texttt{jpg}, \texttt{png}, \texttt{bmp},$\dots$ Cependant, il est préférable d'utiliser un format vectoriel quand c'est possible (\texttt{pdf} or \texttt{eps}).

    \begin{figure}[H]
        \centering
        \includegraphics[width=0.5\textwidth]{resources/img/up_maths-info.jpg}
        \noskipcaption{Logo de l'UFR Maths-Info}
        \label{fig:logo-up-mi}
    \end{figure}

    \subsection{Dessin}
    Pour les plus courageux, il est possible de dessiner directement, notamment avec le package \texttt{tikz}. Voici un exemple de graphe dirigé :
    
    \begin{figure}[H]
        \centering
        \begin{tikzpicture}[->,>=stealth,shorten >=1pt,auto,node distance=1.5cm,
                thick,main node/.style={circle,draw,font=\bfseries}]
        \node[main node] (a1) {$a_1$};
        \node[main node] (a2) [right of=a1] {$a_2$};
        \node[main node] (a3) [right of=a2] {$a_3$};
        \node[main node] (a4) [below of=a2] {$a_4$};

        \path[->] (a2) edge (a1) 
            (a3) edge (a2)
            (a4) edge[bend right] (a2)
            (a2) edge[bend right] (a4);
        \end{tikzpicture}
        \noskipcaption{Un graphe dirigé}
        \label{fig:graphe-dirige}
    \end{figure}
    
    Voir \href{http://math.et.info.free.fr/TikZ/}{Ti$k$Z pour l'impatient} pour plus de détails.

    \section{Tables}

    Voici des exemples de tableaux :

    \begin{table}[H]
        \centering
        \begin{tabular}{|r|r|c|l|}
            \hline
            \multicolumn{3}{|l|}{a} & qrs  \\ \hline
             b &  ef &     jkl      & tuvx \\ \hline
            cd & ghi &     mnop     & wyz  \\ \hline
        \end{tabular}
        \caption{Exemple tableau avec une cellule sur plusieurs colonnes}
        \label{tab:multicol_example}
    \end{table}

    \begin{table}[H]
        \centering
        \begin{tabular}{|l|c|r|}
            \hline
            \multirow{3}{2cm}{a} &   b   &    c \\ \cline{2-3}
                                 &  de   &   fg \\ \cline{2-3}
                                 &  hij  &  klm \\ \hline
            nopq                 & rstuv & wxyz \\ \hline
        \end{tabular}
        \caption{Exemple tableau avec une cellule sur plusieurs lignes}
        \label{tab:multirow_example}
    \end{table}

    \section{Théorèmes}
    Le template fournit plusieurs environnements de type \og théorème \fg, notamment \verb+thm+ (théorème), \verb+lem+ (lemme), \verb+prop+ (proposition), \verb+proof+ (démonstration), \verb+defn+ (définition), ou \verb+expl+ (exemple).

    \begin{thm}[Inégalité triangulaire]
        Étant donné un triangle dans un espace euclidien, la somme des longueurs de deux de ses côtés est supérieure ou égale à la longueur du troisième côté.
    \end{thm}

    \begin{proof}
        Soient $a$, $b$ et $c$ les longueurs des côtés d'un triangle dans un espace euclidien, et $\alpha$, $\beta$, $\gamma$ leurs angles opposés respectifs. D'après le théorème de Pythagore généralisé, on a
        \begin{alignat*}{2}
                                  &  & c^2 & = a^2 + b^2 - 2ab \cos\gamma \\
                                  &  &     & \leq a^2 + b^2 + 2ab         \\
                                  &  &     & \leq (a + b)^2               \\
            \Leftrightarrow \quad &  & c   & \leq a + b
        \end{alignat*}
        Par conséquent, dans n'importe quel triangle, la somme des longueurs de deux côtés est toujours supérieure ou égale à la longueur du troisième côté.
    \end{proof}

    \begin{defn}[Graphe dirigé]
    Un graphe dirigé est un couple $G = \langle N, E \rangle$ où $N$ est l'ensemble des noeuds, et $E \subseteq N \times N$ est l'ensemble des arcs.
    \end{defn}
    
    \begin{expl}
    Le graphe $G = \langle N, E\rangle$ avec $N = \{a_1, a_2, a_3, a_4\}$ et $E = \{(a_2, a_1), (a_2, a_4), (a_3, a_2),\linebreak (a_4, a_2)\}$ est montré en Figure~\ref{fig:graphe-dirige}.
    \end{expl}

    \section{Algorithmes et code}
    Il est possible d'afficher des algorithmes et extraits de code dans un document \LaTeX.
    
    \subsection{Algorithmes en pseudo-code}
    
    Voici un exemple d'algorithme écrit avec le package \texttt{algorithm2e}, nous vous invitons à voir sa \href{http://tug.ctan.org/macros/latex/contrib/algorithm2e/doc/algorithm2e.pdf}{documentation} pour plus d'informations. Il est possible de décrire le comportement d'un algorithme en faisant référence aux lignes numérotées, par exemple la ligne~\ref{algo:line-example}.
    
\begin{algorithm}[H]
  \SetAlgoLined
  \KwInput{une liste d'entiers $l$}\;
  \KwResult{le plus grand élément de $l$ }
  $tmpMax = l[0]$\;
  $i = 1$\;
  \While{$i < longueur(l)$}{
    \If{$l[i] > tmpMax$}{
      $tmpMax = l[i]$\;
    }
    $i = i + 1$\; \label{algo:line-example}
  }
  return $tmpMax$\;
  \caption{Recherche de l'élément maximal d'une liste}
\end{algorithm}

    \subsection{Listings : extraits de code source}

    Le package \texttt{sleek-listings} permet d'afficher des extraits de code source, en respectant la coloration syntaxique pour différents langages : \texttt{c}, \texttt{cpp}, \texttt{matlab}, \texttt{python} and \texttt{java} are implemented, with basic color-maps. Il est possible de personnaliser le style des listings pour chacun de ces langages :
    \begin{itemize}
        \item \texttt{\tbs{}NumberStyle\{stylename\}} crée un style \texttt{stylenameNumber} avec les lignes numérotées ;
        \item \texttt{\tbs{}FrameStyle\{stylename\}} crée un style \texttt{stylenameFrame} avec un cadre autour du listing ;
        \item \texttt{\tbs{}FrameTBStyle\{stylename\}} crée un style \texttt{stylenameFrameTB} avec une ligne au dessus et en dessous du listing;
        \item \texttt{\tbs{}FrameNumberStyle\{stylename\}} et \texttt{\tbs{}FrameTBNumberStyle\{stylename\}} combinent les lignes numérotées et le cadre/les lignes.
    \end{itemize}

    Par exemple, la commande \texttt{\tbs{}FrameTBStyle\{python\}} crée un nouveau style appelé \texttt{pythonFrameTB}, qui donne le rendu suivant pour du code \texttt{Python} :

    \FrameTBStyle{python}
    \begin{lstlisting}[style=pythonFrameTB, gobble=4]
    import numpy as np # Unnecessary import

    a, b = 69., .420

    def f(a: float, b: float) -> float:
        r"""
        Sum two numbers

        Parameters
        ----------
        a: first number
        b: second number

        Returns
        -------
        the sum of 'a' and 'b'
        """

        return a + b

    c = f(a, b)

    print('{:f} + {:f} equals {:f}'.format(a, b, c))
    \end{lstlisting}

    \section{Bibliographie}

Nous recommandons l'usage de \href{https://fr.wikipedia.org/wiki/BibTeX}{BibTeX} pour la gestion de la bibliographie. Ajoutez les entrées correspondant à vos références dans le fichier \verb+resources/bib/references.bib+. Vous pouvez obtenir les entrées BibTeX sur \href{https://scholar.google.com}{Google Scholar} ou \href{https://dblp.uni-trier.de}{DBLP}. La commande \verb+\cite+ vous permet de citer une (ou plusieurs) référence(s) dans le document, par exemple \cite{pakin2020comprehensive,RusselNorvig}.

\chapter{Conclusion}
La conclusion de votre rapport doit brièvement résumer ce que vous avez fait, en mettant en lumière les points forts et les points faibles de votre travail. Enfin, il faut décrire les perspectives de poursuite qui peuvent être envisagées.

    \printbibliography[heading=bibintoc,title={Références}]

    \appendix

    \chapter{Tables}
    Les chapitres d'annexes sont situés après l'utilisatoin de la commande \verb+\appendix+, et se structurent comme des chapitres normaux.
    
    \begin{table}[h]
        \centering
        \begin{tabular}{>{\ttfamily\tbs{}}ll}
            \toprule
            emph\{abcABC123\} & \emph{abcABC123} \\
            bfseries\{abcABC123\} & \bfseries{abcABC123} \\
            itshape\{abcABC123\} & \itshape{abcABC123} \\
            lowercase\{abcABC123\} & \lowercase{abcABC123} \\
            normalfont\{abcABC123\} & \normalfont{abcABC123} \\
            textbf\{abcABC123\} & \textbf{abcABC123} \\
            textit\{abcABC123\} & \textit{abcABC123} \\
            textsc\{abcABC123\} & \textsc{abcABC123} \\
            textsf\{abcABC123\} & \textsf{abcABC123} \\
            textsl\{abcABC123\} & \textsl{abcABC123} \\
            textsubscript\{abcABC123\} & \textsubscript{abcABC123} \\
            textsuperscript\{abcABC123\} & \textsuperscript{abcABC123} \\
            texttt\{abcABC123\} & \texttt{abcABC123} \\
            underline\{abcABC123\} & \underline{abcABC123} \\
            uppercase\{abcABC123\} & \uppercase{abcABC123} \\
            \bottomrule
        \end{tabular}
        \caption{Available text fonts in \LaTeX{}.}
        \label{tab:text_fonts}
    \end{table}

    \begin{table}[h]
        \centering
        \begin{tabular}{>{\ttfamily\$\tbs{}}l<{\$}l}
            \toprule
            mathcal\{abcABC123\} & $\mathcal{abcABC123}$ \\
            mathit\{abcABC123\} & $\mathit{abcABC123}$ \\
            mathnormal\{abcABC123\} & $\mathnormal{abcABC123}$ \\
            mathrm\{abcABC123\} & $\mathrm{abcABC123}$ \\
            mathbb\{abcABC123\} & $\mathbb{abcABC123}$ \\
            mathfrak\{abcABC123\} & $\mathfrak{abcABC123}$ \\
            \bottomrule
        \end{tabular}
        \caption{Available math fonts in \LaTeX{} and AMS.}
        \label{tab:math_fonts}
    \end{table}



    \begin{table}[h]
        \centering
        \begin{tabular}{ll}
            \toprule
            \textbf{Expression} & \textbf{Description} \\
            \midrule
            \texttt{\tbs{}arabic*} & Arabic numbers (1, 2, 3, ...) \\
            \texttt{\tbs{}alph*} & Lowercase letters (a, b, c, ...) \\
            \texttt{\tbs{}Alph*} & Uppercase letters (A, B, C, ...) \\
            \texttt{\tbs{}roman*} & Lowercase Roman numerals (i, ii, iii, ...) \\
            \texttt{\tbs{}Roman*} & Lowercase Roman numerals (I, II, III, ...) \\
            \bottomrule
        \end{tabular}
        \caption{Special expressions for the label of \texttt{enumerate} environments.}
        \label{tab:enumerate_special_expressions}
    \end{table}

\end{document}
