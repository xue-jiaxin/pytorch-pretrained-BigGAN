\documentclass[a4paper, 12pt]{report}

%%%%%%%%%%%%
% Packages %
%%%%%%%%%%%%

\usepackage[french]{babel}
\usepackage[noheader]{packages/sleek}
\usepackage{packages/sleek-title}
\usepackage[french]{packages/sleek-theorems}
\usepackage{packages/sleek-listings}
\usepackage{tikz}
\usepackage[french,linesnumbered,lined]{algorithm2e}
\usepackage{movie15}
\usepackage{graphicx}

\SetKwInput{KwResult}{R\'esultat}
\SetKw{KwInput}{Entr\'ees}
%%%%%%%%%%%%%%
% Title-page %
%%%%%%%%%%%%%%

\logo{./resources/img/up_maths-info.jpg}
\institute{Université de Paris}
\faculty{UFR Mathématiques et Informatique}
\title{Projet TER - StyleGANov'}
\subtitle{Master 1 Vision Machine Intelligente}
%\subtitle{Master 1 Vision et Machine Intelligente}
\author{Jiaxin \textsc{XUE} -- Billal \textsc{IHADDADEN}\\
{\small Encadré par Olivier \textsc{RISSER-MAROIX}}}
\date{Année universitaire 2020-2021}

%%%%%%%%%%%%%%%%https://www.overleaf.com/project/6012bd305aee4065605e8f1d
% Bibliography %
%%%%%%%%%%%%%%%%
\usepackage{biblatex}
\addbibresource{references.bib}

%%%%%%%%%%
% Others %
%%%%%%%%%%

\lstdefinestyle{latex}{
    language=TeX,
    style=default,
    %%%%%
    commentstyle=\ForestGreen,
    keywordstyle=\TrueBlue,
    stringstyle=\VeronicaPurple,
    emphstyle=\TrueBlue,
    %%%%%
    emph={LaTeX, usepackage, textit, textbf, textsc}
}

\FrameTBStyle{latex}

\def\tbs{\textbackslash}

%%%%%%%%%%%%
% Document %
%%%%%%%%%%%%

\begin{document}
    \maketitle

    \romantableofcontents

    
    \chapter{Introduction}

Le jugement de la similarité d'images par l'humain s'appuie sur beaucoup de choses notamment les éléments de la scène et les aspects culturels. Pour nous les humains il nous est facile de juger de la similarité entre deux images, cependant la prédiction de la similarité perceptive humaine est un sujet de recherche difficile. Le processus visuel sous-jacent à cet aspect de la vision humaine fait appel à plusieurs niveaux différents d'analyse visuelle (formes, objets, texture, disposition, couleur…etc). Dans le cas de ce projet la similarité purement visuelle sera traitée sans prendre en compte la sémantique. Le transfert de style et la génération d'image par des architectures modernes de réseaux de neurones (VAE, ADalN, PIX2PIX, CycleGAN, BigGAN…etc). La première partie consiste à expérimenter la génération d'image avec pix2pix en utilisant les images de l'ensemble de données PASCAL avec et sans l'arrière-plan. Ensuite Nous allons expérimenter le BigGAN pré entrainé sur l'ensemble de données d'ImageNet.
    % in documenet
    %\includemovie{2cm}{2cm}{cat2dog.gif}
    %\includegraphics{cat2dog.gif}
    
    %\chapter{État de l'art}
    %Un chapitre dédié à l'état de l'art doit décrire les concepts et méthodes déjà existant(e)s, en lien avec le travail que vous avez réalisé.
    
    \chapter{État de l'art}

Un chapitre dédié à l'état de l'art doit décrire les concepts et méthodes déjà existant(e)s, en lien avec le travail que vous avez réalisé.

\section{La similarité visuelle en science cognitives}

Les objets peuvent être caractérisés selon un grand nombre de critères possibles, mais certaines caractéristiques sont plus utiles que d’autres pour donner un sens aux objets qui nous entourent. Dans [5] ils ont développé un modèle informatique de jugements de similarité pour les images du monde réel de 1854 objets.

\section{La similarité visuelle et l’apprentissage profond}

Les progrès récents des réseaux de neurones artificiels ont révolutionné la vision par ordinateur, mais ces systèmes de conception sont toujours surpassés par les humains, dans [1] ils ont comparé la perception des objets par le cerveau humain et par les machines (certain nombre de modèles informatiques, par exemple : Tal, Gabor, Hog, split-half…etc.).  Ils ont recueilli un vaste ensemble de données comprenant 26 675 mesures de dissimilarité perçue pour 2801 objets visuels chez 269 sujets humains.  Afin de mesurer la dissimilarité chez les humains, ils ont demandé de localiser une bille étrange dans un tableau contenant un objet parmi de multiples occurrences de l’autre. La réciproque du temps de recherche visuelle a été considérée comme une estimation de la dissimilarité perçue. Cette mesure se comporte comme une distance mathématique, elle présente une somme linéaire de plusieurs caractéristiques, elle explique la catégorisation visuelle rapide et est fortement corrélée avec les évaluations subjectives de la dissimilarité. Ils ont testé l’ensemble de mesures sur des modèles de calcul très répandu. Le meilleur modèle était un CNN mais il a été surpassé par la combinaison de tous les autres modèles. Leur conclusion était que tous les modèles informatiques montrent des modèles similaires d’écart par rapport à la perception humaine.
 

Dans [2] ils voulaient savoir est-ce que l’apprentissage automatique peut expliquer les jugements humains de similarité de forme d’objets visuels. Alors ils ont analysé la performance des systèmes d’apprentissage métrique (distance ou similarité) y compris les DNN, sur un nouvel ensemble de données de jugement de similarité de forme d’objet visuel humain. Contrairement aux autres études ou ils demandaient aux participants de juger de la similarité lorsque les objets ou les scènes étaient rendus à partir d’un seul point de vue, eux ils ont utilisé un rendu à partir de plusieurs vues et ils ont demandé aux participants de juger de la similarité de forme de manière variable. Ils ont trouvé que les DNN ne parviennent pas à expliquer les données expérimentales, mais une métrique entraînée avec une représentation variable basée sur des parties produit un bon ajustement, ils ont aussi constaté que même si les DNN puissent apprendre à extraire la représentation basée sur les parties et devrait être capable d’apprendre à modéliser leurs données. Les réseaux entraînés avec une fonction “triplet loss” basée sur le jugement de similarité ne donne pas un bon résultat. Le mauvais résultat du DNN est causé par la non-convexité du problème d’optimisation dans l’espace des poids du réseau. Ils concluent que l’insensibilité du point de vue est un aspect critique de la perception de la forme visuelle humaine, et que les réseaux de neurones et d’autres méthodes d’apprentissage automatique devront apprendre des représentations insensibles au point de vue afin de rendre compte des jugements de similarité de forme d’objets visuels des humains.

La comparaison des représentations formées par les DNN avec celles utilisées par les humains est un défi, car les représentations psychologiques humaines ne peuvent pas être observées directement. Dans [3] ils ont évalué et proposé une amélioration de la correspondance entre les DNN et les représentations humaines. Leur approche consiste à résoudre le problème de comparaison en exploitant la relation étroite entre représentation et similarité, ce qui veut dire que pour chaque fonction de similarité sur un ensemble de paires de points de données correspond à une représentation implicite de ces points. Ce qui offre une base empirique pour la première évaluation de DNN en tant qu’une approximation des représentations psychologiques humaines.
  
Dans [4] ils ont démontré leur méthode sur le jeu de données CUB-200-2011 et Stanford Cars en appliquant leur architecture du DNN ProtoPNet. Leur expérience a montré que l’architecture de DNN qu’ils ont créé pouvait atteindre une précision comparable à avec ses analogues. Et lorsque plusieurs ProtoPNet sont combinés en un réseau plus vaste, celui-ci peut atteindre une précision équivalente à celle de certains des modèles profonds les plus performants. De plus, leur modèle offre un niveau d’interprétabilité qui est absent dans les modèles existants. 


\section{Les GANs et le transfert du style}
    
    \chapter{Méthodes étudiées}
Depuis que Ian Goodfellow a proposé le GAN (Generative Adversarial Network) en 2014, la recherche sur le GAN est très active. Diverses variantes du GAN ne cessent d'apparaître. Yann LeCun a même déclaré que le GAN était \og{}adversarial training is the coolest thing since sliced bread.\fg{} en 2016. Nous nous donc focaliserons dans un sur la génération de paires d’images visuellement similaires via des GANs. 

\section{Generative Adversarial Network}

Generative Adversarial Network (GAN) est une méthode d'apprentissage non supervisé, consistant à faire jouer deux réseaux neuronaux l'un contre l'autre. Cette méthode a été proposée en 2014 par Ian Goodfellow\cite{goodfellow2014generative}. Les GANs se composent d'un réseau génératif (générateur) et d'un réseau discriminatif (discriminateur). Dans entraîner GAN, le générateur génère un échantillon (ex. une image), tandis que son adversaire, le discriminateur essaie de distinguer si cet échantillon est réel ou non. Le but du générateur est de pouvoir tromper le discriminateur autant que possible. Les deux réseaux jouent l'un contre l'autre et ajustent constamment leurs paramètres, dans le but ultime de rendre le discriminateur incapable de déterminer si la sortie du réseau génératif est fausse. 

Un des principales contributions de GAN est la stratégie de training qui rend le discriminateur pouvoir reconnaitre la distribution appris par générateur et la distribution réelle. Et, face à des problèmes tels que la difficulté de training, il y a de nombreux améliorations et développements sur l'original tels que c-GAN, cycle-GAN, styleGAN.

\section{pixp2ix} 
pix2pix modèle propose un cadre général pour la tâche de la traduction d'image à image en combinant cGAN pour réaliser la traduction d'image du domaine source au domaine cible. Le réseau apprend non seulement la correspondance entre l'image d'entrée à l'image de sortie, mais apprennent également une fonction de perte pour entraîner cette correspondance. Cela permet d'appliquer la même approche générique à des problèmes qui, traditionnellement nécessiteraient des formulations de perte très différentes\cite{isola2017image}. 

En termes de générateur, pix2pix utilise Unet comme générateur, en considérant que l'aspect de surface des images d'entrée et de sortie doit être différent alors que la structure de base doit être similaire, et que pour la tâche de traduction d'image, l'entrée et la sortie doivent partager certaines informations de base (ex. contours), ils appliquent donc une connexion de layer-skipping comme l'approche de connexion dans Unet.

En termes de discriminateur, l'auteur a crée PatchGAN comme discriminateur, au lieu de  discriminateur qui base sur les distance traditionnelles L1, L2 dans les travaux précédents. L'idée de PatchGAN est de diviser l'image en partie avec over-lapping, ensuite juger la vérité ou la fausseté de chaque patch séparément. Les auteurs concluent en disant que le PatchGAN proposé peut être considéré comme une autre forme de perte de texture ou de perte de style.

Avec pix2pix, si nous fournissons des pairs d'images similaires, nous pouvons former un modèle qui prend une image d'entrée et nous génère une image similaire.


\section{BigGAN et Espace de Latent}

Grâce à l'avènement des GANs, les algorithmes de modélisation générative d'images ont fait de grands progrès ces dernières années pour générer des images réalistes et diverses. Cependant, la génération d'images diverses et à haute résolution à partir d'ensembles de données complexes comme ImageNet reste une tâche difficile. BigGAN est le modèle génératif le plus grand et le mieux entrainé à ce jour. Les images générées par BigGAN atteignent un niveau de réalisme extrêmement élevé. \cite{brock2018large}

Bien sûr, nous n'avons pas l'intention de former un GAN à partir de zéro, et les modèles de type BigGAN nécessitent d'énormes ressources informatiques pour être formés. Avec l'aide du BigGAN déjà bien formé sur ImageNet, nous considérons essayer de faire varier l'entrée du générateur de manière conditionnelle. Cela implique l'espace latent.

Härkönen, Erik, et al\cite{harkonen2020ganspace} décrit une technique simple pour analyser les réseaux adversariaux génératifs (GAN) et créer des commandes interprétables pour la synthèse d'images, comme le changement de point de vue, le vieillissement, l'éclairage et l'heure de la journée.




    
    \chapter{Méthodes expérimentales}
    
    \chapter{Conclusion et Perspectives}
Pour la première étape nous avons réussis à générer des images grâce à pix2pix nous avons eu 3 modèles différents (uniquement avec les voitures, toutes les images de la dataset PASCAL avec et sans le background), Nous avons obtenu des résultats pas très satisfaisant ou les images générées sont presque carrément différentes. Ensuite pour la deuxième étape nous avons expérimenté le BigGan  pré entrainé avec la dataset imageNet ou nous avons changé un peu  l'algorithme en modifiant l'espace Latent pour la première phase de cette expérimentation nous avons changé que le bruit  ensuit dans la deuxième phase nous avons réussis à changer le vecteur de classe et le bruit afin de comparer les résultats. Nous avons remarqué que la génération d'image après avoir modifié que le bruit en choisissant des classes de qui se ressemble juste un peu, notre algorithme permet de générer des images similaire (ex. chat vers chien) par contre quand nous choisissant deux classes qui sont complètement différentes notre algorithme génère une image complétement dissimilaire à notre image but.

La similarité est vraiment un monde vaste, nous avons réussis à obtenir des résultats qui sont assez satisfaisants, mais il existe d'autres approches. Afin d'améliorer les résultats il faudrait essayer de générer plusieurs exemple en modifiant l'espace latent (bruit, classe, bruit et classe) et cela en changeant l'optimiseur et en changeant aussi les poids des loss sémantique et pixels.  A la fin une comparaison devra être faite par des sujets humains sur les images pour choisir les meilleurs poids et le meilleur optimiseur et bien évidement le choix du changement de l'espace latent.
    
    
    
    %\chapter{Contribution}

	\chapter{Conclusion}
	La conclusion de votre rapport doit brièvement résumer ce que vous avez fait, en mettant en lumière les points forts et les points faibles de votre travail. Enfin, il faut décrire les perspectives de poursuite qui peuvent être envisagées.

	%\printbibliography
	\printbibliography[heading=bibintoc,title={Références}]


	

\end{document}
