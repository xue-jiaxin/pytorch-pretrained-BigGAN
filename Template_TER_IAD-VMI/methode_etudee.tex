\chapter{Méthodes étudiées}
Depuis que Ian Goodfellow a proposé le GAN (Generative Adversarial Network) en 2014, la recherche sur le GAN est très active. Diverses variantes du GAN ne cessent d'apparaître. Yann LeCun a même déclaré que le GAN était \og{}adversarial training is the coolest thing since sliced bread.\fg{} en 2016. Nous nous donc focaliserons dans un sur la génération de paires d’images visuellement similaires via des GANs. 

\section{Generative Adversarial Network}

Generative Adversarial Network (GAN) est une méthode d'apprentissage non supervisé, consistant à faire jouer deux réseaux neuronaux l'un contre l'autre. Cette méthode a été proposée en 2014 par Ian Goodfellow\cite{goodfellow2014generative}. Les GANs se composent d'un réseau génératif (générateur) et d'un réseau discriminatif (discriminateur). Dans entraîner GAN, le générateur génère un échantillon (ex. une image), tandis que son adversaire, le discriminateur essaie de distinguer si cet échantillon est réel ou non. Le but du générateur est de pouvoir tromper le discriminateur autant que possible. Les deux réseaux jouent l'un contre l'autre et ajustent constamment leurs paramètres, dans le but ultime de rendre le discriminateur incapable de déterminer si la sortie du réseau génératif est fausse. Un des principales contributions de GAN est la stratégie de training qui rend le discriminateur pouvoir reconnaitre la distribution appris par générateur et la distribution réelle. Et, face à des problèmes tels que la difficulté de training, il y a de nombreux améliorations et développements sur l'original tels que c-GAN

\section{Pix2Pix}

\section{BigGAN et espace de latent}